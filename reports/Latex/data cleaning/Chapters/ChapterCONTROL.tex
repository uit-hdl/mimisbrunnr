%*****************************************
\chapter{Data Control}\label{ch:control}
%*****************************************

In this chapter we study each of the variables in order to find anomalies or inconsistencies that need clarification or better definitions.

\section{Summary of findings}

\section{SA}

First let see that the grow process is valid.\vspace{3 mm}

How many rows do we have that grow something for NASAL and THROAT? \vspace{3 mm}

\input{../img/results/all/AbsBarplot_completeTable_NasalGrowth_.tex}

\input{../img/results/all/AbsBarplot_completeTable_ThroatGrowth_.tex}

We have about 1000 grows in each. Those in which nothinig grow, or we don't know the awnser, is not applicable, etc... are removed. So these automatically a NEGATIVE for SA carrier.\vspace{3 mm}

For those samples in which something grew, we need to do two more analysis.\vspace{3 mm}

The first one is the coagulase test. This consist into use the aglutination process of the the SA. If there is SA in our sample, the SA will merge toguether into a blob and fall down the test tube. This is considered a positive coagulase test. In this case positive for coagulase means positive for SA. But beware that there are other bacterias which a negative coagulase test will indicate a presence of that bacteria; for example for S. epidermidis or S. saprophyticus. Furthermore, some SA types don't pass a positive coagulase test REF! ; but nevertheless, in our case, we consider they do. \vspace{3 mm}


REF: Ryan KJ, Ray CG (editors) (2004). Sherris Medical Microbiology (4th ed.). McGraw Hill. ISBN 0-8385-8529-9.
REF: PreTest, Surgery, 12th ed., p.88


 So if something grow, we need to see \vspace{3 mm}